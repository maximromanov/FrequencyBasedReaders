\subsection*{Biography, Rank \# 2}

\tiny \begin{flushright}\texttt{Taʾrīḫ al-islām (000000002-023270-000.txt)}\end{flushright}

\begin{arab}[novoc]
\large \noindent الحسن بن محمد بن موسى بن إسحاق بن موسى، أبو علي الأنصاري. [الوفاة: 342 ه] سمع جده موسى، وابن أبي الدنيا، والمبرد، وغيرهم. وعنه القاضي أبو القاسم بن أبي عمرو، ومحمد بن أحمد بن أبي عون شيخا الخطيب. وثقه الخطيب، وقال: توفي في ذي الحجة.
\end{arab}

\normalsize
\begin{center}\textit{Vocabulary (by frequencies)}\end{center}
\footnotesize
\begin{multicols}{3}

\noindent
1) \mbox{\textarab{بن}} ---  son;\\
2) \mbox{\textarab{في}} ---  in;\\
3) \mbox{\textarab{أبو}} ---  father;\\
6) \mbox{\textarab{محمد}} ---  Muḥammad, \textit{name};\\
9) \mbox{\textarab{الوفاة}} ---  death;\\
9) \mbox{\textarab{ه}} ---  abbr. of \textit{hiǧrī}, Islamic lunar calendar;\\
11) \mbox{\textarab{أحمد}} ---  Aḥmad, \textit{name};\\
12) \mbox{\textarab{وقال}} ---  and [he] said;\\
13) \mbox{\textarab{علي}} ---  ʿAlī, \textit{name};\\
18) \mbox{\textarab{وعنه}} ---  and from him;\\
19) \mbox{\textarab{الحسن}} ---  al-Ḥasan, \textit{name};\\
19) \mbox{\textarab{توفي}} ---  (passive) [he] died, \textit{lit.} ``he was taken [by God]'';\\
19) \mbox{\textarab{سمع}} ---  [he] heard, listened (i.e., studied);\\
35) \mbox{\textarab{القاسم}} ---  al-Qāsim, \textit{name};\\
48) \mbox{\textarab{إسحاق}} ---  Isḥāq (Isaac), \textit{name};\\
64) \mbox{\textarab{عمرو}} ---  ʿAmr, \textit{name};\\
90) \mbox{\textarab{القاضي}} ---  [the] judge, \textit{qāḍī};\\
90) \mbox{\textarab{موسى}} ---  Mūsá (Moses), \textit{name};\\
116) \mbox{\textarab{الخطيب}} ---  [the] orator, Friday preacher; here, \textit{name} of a prominent Ḥadīṯ scholar---al-Ḫaṭīb al-Baġdādī;\\
117) \mbox{\textarab{ذي}} ---  gen. of \textit{ḏū}, owner, possessor, ``that of ...'';\\
129) \mbox{\textarab{وغيرهم}} ---  and others, lit. "and other than them";\\
152) \mbox{\textarab{وثقه}} ---  [he] considered him trustworthy (technical term);\\
161) \mbox{\textarab{الأنصاري}} ---  al-Anṣārī, \textit{nisbaŧ} denoting descendence from the Anṣār, ``the Helpers of the Prophet'';\\
259) \mbox{\textarab{الحجة}} ---  here, a part of \textit{ḏū-l-ḥiǧǧaŧ}, the 12th month of the Islamic lunar calendar;\\
301) \mbox{\textarab{الدنيا}} ---  ``this world'', ``this life''; here, a part of the name of Ibn Abī-l-Dunyā, a prominent early scholar;\\
301) \mbox{\textarab{جده}} ---  his grandfather;\\
454) \mbox{\textarab{شيخا}} ---  two \textit{šayḫ}s, two teachers;\\
952) \mbox{\textarab{والمبرد}} ---  al-Mubarrad, a prominent Ḥadīṯ scholar;\\ % "%s) \mbox{\textarab{%s}} --- %s;\\ \n" % (rank, token, transl)
\end{multicols}

\normalsize
\noindent \textbf{Grammar:} For \textarab{شيخا الخطيب}, review dual and its behavior in \textit{iḍāfaŧ} (\textsc{Thackston}, IKCA, L4-§8; for \textit{iḍāfaŧ}, see \textsc{Thackston}, IKCA, L4-§8). For \textarab{وثقه}, see Factitive verbs of Form II: \textsc{Thackston}, IKCA, L27-§64. For \textarab{توفي}, see Reflexive/Medio-passive verbs, Form V: \textsc{Thackston}, IKCA, L29-§67.

\normalsize
\noindent \textbf{Culture:} Traditional Arab name is quite complex and includes up to six different parts. Here we find person's ``first name" (\textit{ism}), which is al-Ḥasan. Then we find his ``genealogy" (\textit{nasab})---the male names connected with \textit{bn}---the name of his father, his grandfather, his great grandfather and his great great grandfather; \textit{nasab} can go back as far as to the time of the Prophet. Then we find \textit{kunyaŧ}, ``the father of ...''---Abū ʿAlī. And the last one is \textit{nisbaŧ}---al-Anṣārī: this type of names asserts relationship between a person and some kind of entity. In this case, the person traces his lineage back to the Anṣār, ``the Helpers [of the Prophet],'' who accepted Muḥammad in Yaṯrib as their leader. Yaṯrib later became known as Madīnaŧ al-Nabī, or simply al-Madīnaŧ (Medina).

\normalsize
\begin{center}\textit{Frequency report}\end{center}
\tiny
\begin{multicols}{5}
\noindent
1) \mbox{\textarab{بن}} (193,727);\\
2) \mbox{\textarab{في}} (48,775);\\
3) \mbox{\textarab{أبو}} (42,002);\\
6) \mbox{\textarab{محمد}} (37,422);\\
8) \mbox{\textarab{أبي}} (32,466);\\
9) \mbox{\textarab{الوفاة}} (30,980);\\
9) \mbox{\textarab{ه}} (30,956);\\
11) \mbox{\textarab{أحمد}} (19,940);\\
12) \mbox{\textarab{وقال}} (17,961);\\
13) \mbox{\textarab{علي}} (17,171);\\
18) \mbox{\textarab{وعنه}} (10,606);\\
19) \mbox{\textarab{الحسن}} (10,316);\\
19) \mbox{\textarab{توفي}} (10,311);\\
19) \mbox{\textarab{سمع}} (10,138);\\
30) \mbox{\textarab{وابن}} (7,550);\\
31) \mbox{\textarab{ومحمد}} (7,533);\\
35) \mbox{\textarab{القاسم}} (6,637);\\
48) \mbox{\textarab{إسحاق}} (4,509);\\
64) \mbox{\textarab{عمرو}} (3,677);\\
90) \mbox{\textarab{القاضي}} (3,104);\\
90) \mbox{\textarab{موسى}} (3,104);\\
116) \mbox{\textarab{الخطيب}} (2,398);\\
117) \mbox{\textarab{ذي}} (2,376);\\
129) \mbox{\textarab{وغيرهم}} (2,101);\\
152) \mbox{\textarab{وثقه}} (1,838);\\
161) \mbox{\textarab{الأنصاري}} (1,773);\\
259) \mbox{\textarab{الحجة}} (1,143);\\
301) \mbox{\textarab{الدنيا}} (973);\\
301) \mbox{\textarab{جده}} (973);\\
428) \mbox{\textarab{عون}} (611);\\
454) \mbox{\textarab{شيخا}} (562);\\
952) \mbox{\textarab{والمبرد}} (17);\\ % "%s) \mbox{\textarab{%s}} (%s);\\ \n" % (rank, token, freq)
\end{multicols}